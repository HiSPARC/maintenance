% Generated by Sphinx.
\def\sphinxdocclass{report}
\documentclass[a4paper,11pt,english]{sphinxmanual}
\usepackage[utf8]{inputenc}
\DeclareUnicodeCharacter{00A0}{\nobreakspace}
\usepackage[T1]{fontenc}
\usepackage{babel}
\usepackage{times}
\usepackage[Bjarne]{fncychap}
\usepackage{longtable}
\usepackage{sphinx}
\usepackage{multirow}


\title{HiSPARC Station Problem and Solutions Documentation}
\date{December 18, 2012}
\release{0.1 alpha}
\author{HiSPARC team}
\newcommand{\sphinxlogo}{\includegraphics{logo.pdf}\par}
\renewcommand{\releasename}{Release}
\makeindex

\makeatletter
\def\PYG@reset{\let\PYG@it=\relax \let\PYG@bf=\relax%
    \let\PYG@ul=\relax \let\PYG@tc=\relax%
    \let\PYG@bc=\relax \let\PYG@ff=\relax}
\def\PYG@tok#1{\csname PYG@tok@#1\endcsname}
\def\PYG@toks#1+{\ifx\relax#1\empty\else%
    \PYG@tok{#1}\expandafter\PYG@toks\fi}
\def\PYG@do#1{\PYG@bc{\PYG@tc{\PYG@ul{%
    \PYG@it{\PYG@bf{\PYG@ff{#1}}}}}}}
\def\PYG#1#2{\PYG@reset\PYG@toks#1+\relax+\PYG@do{#2}}

\expandafter\def\csname PYG@tok@gd\endcsname{\def\PYG@tc##1{\textcolor[rgb]{0.63,0.00,0.00}{##1}}}
\expandafter\def\csname PYG@tok@gu\endcsname{\let\PYG@bf=\textbf\def\PYG@tc##1{\textcolor[rgb]{0.50,0.00,0.50}{##1}}}
\expandafter\def\csname PYG@tok@gt\endcsname{\def\PYG@tc##1{\textcolor[rgb]{0.00,0.25,0.82}{##1}}}
\expandafter\def\csname PYG@tok@gs\endcsname{\let\PYG@bf=\textbf}
\expandafter\def\csname PYG@tok@gr\endcsname{\def\PYG@tc##1{\textcolor[rgb]{1.00,0.00,0.00}{##1}}}
\expandafter\def\csname PYG@tok@cm\endcsname{\let\PYG@it=\textit\def\PYG@tc##1{\textcolor[rgb]{0.25,0.50,0.56}{##1}}}
\expandafter\def\csname PYG@tok@vg\endcsname{\def\PYG@tc##1{\textcolor[rgb]{0.73,0.38,0.84}{##1}}}
\expandafter\def\csname PYG@tok@m\endcsname{\def\PYG@tc##1{\textcolor[rgb]{0.13,0.50,0.31}{##1}}}
\expandafter\def\csname PYG@tok@mh\endcsname{\def\PYG@tc##1{\textcolor[rgb]{0.13,0.50,0.31}{##1}}}
\expandafter\def\csname PYG@tok@cs\endcsname{\def\PYG@tc##1{\textcolor[rgb]{0.25,0.50,0.56}{##1}}\def\PYG@bc##1{\setlength{\fboxsep}{0pt}\colorbox[rgb]{1.00,0.94,0.94}{\strut ##1}}}
\expandafter\def\csname PYG@tok@ge\endcsname{\let\PYG@it=\textit}
\expandafter\def\csname PYG@tok@vc\endcsname{\def\PYG@tc##1{\textcolor[rgb]{0.73,0.38,0.84}{##1}}}
\expandafter\def\csname PYG@tok@il\endcsname{\def\PYG@tc##1{\textcolor[rgb]{0.13,0.50,0.31}{##1}}}
\expandafter\def\csname PYG@tok@go\endcsname{\def\PYG@tc##1{\textcolor[rgb]{0.19,0.19,0.19}{##1}}}
\expandafter\def\csname PYG@tok@cp\endcsname{\def\PYG@tc##1{\textcolor[rgb]{0.00,0.44,0.13}{##1}}}
\expandafter\def\csname PYG@tok@gi\endcsname{\def\PYG@tc##1{\textcolor[rgb]{0.00,0.63,0.00}{##1}}}
\expandafter\def\csname PYG@tok@gh\endcsname{\let\PYG@bf=\textbf\def\PYG@tc##1{\textcolor[rgb]{0.00,0.00,0.50}{##1}}}
\expandafter\def\csname PYG@tok@ni\endcsname{\let\PYG@bf=\textbf\def\PYG@tc##1{\textcolor[rgb]{0.84,0.33,0.22}{##1}}}
\expandafter\def\csname PYG@tok@nl\endcsname{\let\PYG@bf=\textbf\def\PYG@tc##1{\textcolor[rgb]{0.00,0.13,0.44}{##1}}}
\expandafter\def\csname PYG@tok@nn\endcsname{\let\PYG@bf=\textbf\def\PYG@tc##1{\textcolor[rgb]{0.05,0.52,0.71}{##1}}}
\expandafter\def\csname PYG@tok@no\endcsname{\def\PYG@tc##1{\textcolor[rgb]{0.38,0.68,0.84}{##1}}}
\expandafter\def\csname PYG@tok@na\endcsname{\def\PYG@tc##1{\textcolor[rgb]{0.25,0.44,0.63}{##1}}}
\expandafter\def\csname PYG@tok@nb\endcsname{\def\PYG@tc##1{\textcolor[rgb]{0.00,0.44,0.13}{##1}}}
\expandafter\def\csname PYG@tok@nc\endcsname{\let\PYG@bf=\textbf\def\PYG@tc##1{\textcolor[rgb]{0.05,0.52,0.71}{##1}}}
\expandafter\def\csname PYG@tok@nd\endcsname{\let\PYG@bf=\textbf\def\PYG@tc##1{\textcolor[rgb]{0.33,0.33,0.33}{##1}}}
\expandafter\def\csname PYG@tok@ne\endcsname{\def\PYG@tc##1{\textcolor[rgb]{0.00,0.44,0.13}{##1}}}
\expandafter\def\csname PYG@tok@nf\endcsname{\def\PYG@tc##1{\textcolor[rgb]{0.02,0.16,0.49}{##1}}}
\expandafter\def\csname PYG@tok@si\endcsname{\let\PYG@it=\textit\def\PYG@tc##1{\textcolor[rgb]{0.44,0.63,0.82}{##1}}}
\expandafter\def\csname PYG@tok@s2\endcsname{\def\PYG@tc##1{\textcolor[rgb]{0.25,0.44,0.63}{##1}}}
\expandafter\def\csname PYG@tok@vi\endcsname{\def\PYG@tc##1{\textcolor[rgb]{0.73,0.38,0.84}{##1}}}
\expandafter\def\csname PYG@tok@nt\endcsname{\let\PYG@bf=\textbf\def\PYG@tc##1{\textcolor[rgb]{0.02,0.16,0.45}{##1}}}
\expandafter\def\csname PYG@tok@nv\endcsname{\def\PYG@tc##1{\textcolor[rgb]{0.73,0.38,0.84}{##1}}}
\expandafter\def\csname PYG@tok@s1\endcsname{\def\PYG@tc##1{\textcolor[rgb]{0.25,0.44,0.63}{##1}}}
\expandafter\def\csname PYG@tok@gp\endcsname{\let\PYG@bf=\textbf\def\PYG@tc##1{\textcolor[rgb]{0.78,0.36,0.04}{##1}}}
\expandafter\def\csname PYG@tok@sh\endcsname{\def\PYG@tc##1{\textcolor[rgb]{0.25,0.44,0.63}{##1}}}
\expandafter\def\csname PYG@tok@ow\endcsname{\let\PYG@bf=\textbf\def\PYG@tc##1{\textcolor[rgb]{0.00,0.44,0.13}{##1}}}
\expandafter\def\csname PYG@tok@sx\endcsname{\def\PYG@tc##1{\textcolor[rgb]{0.78,0.36,0.04}{##1}}}
\expandafter\def\csname PYG@tok@bp\endcsname{\def\PYG@tc##1{\textcolor[rgb]{0.00,0.44,0.13}{##1}}}
\expandafter\def\csname PYG@tok@c1\endcsname{\let\PYG@it=\textit\def\PYG@tc##1{\textcolor[rgb]{0.25,0.50,0.56}{##1}}}
\expandafter\def\csname PYG@tok@kc\endcsname{\let\PYG@bf=\textbf\def\PYG@tc##1{\textcolor[rgb]{0.00,0.44,0.13}{##1}}}
\expandafter\def\csname PYG@tok@c\endcsname{\let\PYG@it=\textit\def\PYG@tc##1{\textcolor[rgb]{0.25,0.50,0.56}{##1}}}
\expandafter\def\csname PYG@tok@mf\endcsname{\def\PYG@tc##1{\textcolor[rgb]{0.13,0.50,0.31}{##1}}}
\expandafter\def\csname PYG@tok@err\endcsname{\def\PYG@bc##1{\setlength{\fboxsep}{0pt}\fcolorbox[rgb]{1.00,0.00,0.00}{1,1,1}{\strut ##1}}}
\expandafter\def\csname PYG@tok@kd\endcsname{\let\PYG@bf=\textbf\def\PYG@tc##1{\textcolor[rgb]{0.00,0.44,0.13}{##1}}}
\expandafter\def\csname PYG@tok@ss\endcsname{\def\PYG@tc##1{\textcolor[rgb]{0.32,0.47,0.09}{##1}}}
\expandafter\def\csname PYG@tok@sr\endcsname{\def\PYG@tc##1{\textcolor[rgb]{0.14,0.33,0.53}{##1}}}
\expandafter\def\csname PYG@tok@mo\endcsname{\def\PYG@tc##1{\textcolor[rgb]{0.13,0.50,0.31}{##1}}}
\expandafter\def\csname PYG@tok@mi\endcsname{\def\PYG@tc##1{\textcolor[rgb]{0.13,0.50,0.31}{##1}}}
\expandafter\def\csname PYG@tok@kn\endcsname{\let\PYG@bf=\textbf\def\PYG@tc##1{\textcolor[rgb]{0.00,0.44,0.13}{##1}}}
\expandafter\def\csname PYG@tok@o\endcsname{\def\PYG@tc##1{\textcolor[rgb]{0.40,0.40,0.40}{##1}}}
\expandafter\def\csname PYG@tok@kr\endcsname{\let\PYG@bf=\textbf\def\PYG@tc##1{\textcolor[rgb]{0.00,0.44,0.13}{##1}}}
\expandafter\def\csname PYG@tok@s\endcsname{\def\PYG@tc##1{\textcolor[rgb]{0.25,0.44,0.63}{##1}}}
\expandafter\def\csname PYG@tok@kp\endcsname{\def\PYG@tc##1{\textcolor[rgb]{0.00,0.44,0.13}{##1}}}
\expandafter\def\csname PYG@tok@w\endcsname{\def\PYG@tc##1{\textcolor[rgb]{0.73,0.73,0.73}{##1}}}
\expandafter\def\csname PYG@tok@kt\endcsname{\def\PYG@tc##1{\textcolor[rgb]{0.56,0.13,0.00}{##1}}}
\expandafter\def\csname PYG@tok@sc\endcsname{\def\PYG@tc##1{\textcolor[rgb]{0.25,0.44,0.63}{##1}}}
\expandafter\def\csname PYG@tok@sb\endcsname{\def\PYG@tc##1{\textcolor[rgb]{0.25,0.44,0.63}{##1}}}
\expandafter\def\csname PYG@tok@k\endcsname{\let\PYG@bf=\textbf\def\PYG@tc##1{\textcolor[rgb]{0.00,0.44,0.13}{##1}}}
\expandafter\def\csname PYG@tok@se\endcsname{\let\PYG@bf=\textbf\def\PYG@tc##1{\textcolor[rgb]{0.25,0.44,0.63}{##1}}}
\expandafter\def\csname PYG@tok@sd\endcsname{\let\PYG@it=\textit\def\PYG@tc##1{\textcolor[rgb]{0.25,0.44,0.63}{##1}}}

\def\PYGZbs{\char`\\}
\def\PYGZus{\char`\_}
\def\PYGZob{\char`\{}
\def\PYGZcb{\char`\}}
\def\PYGZca{\char`\^}
\def\PYGZam{\char`\&}
\def\PYGZlt{\char`\<}
\def\PYGZgt{\char`\>}
\def\PYGZsh{\char`\#}
\def\PYGZpc{\char`\%}
\def\PYGZdl{\char`\$}
\def\PYGZti{\char`\~}
% for compatibility with earlier versions
\def\PYGZat{@}
\def\PYGZlb{[}
\def\PYGZrb{]}
\makeatother

\begin{document}

\maketitle
\tableofcontents
\phantomsection\label{index::doc}


The HiSPARC software can be downloaded here: \href{http://www.hisparc.nl/downloads/software/}{HiSPARC Installer}

A pdf version of this manual is available here: HiSPARC maintenance manual

For more information about HiSPARC, see the \href{http://www.hisparc.nl/}{HiSPARC Website}.

For questions or suggestions please contact our project coordinator
Surya Bonam (\href{mailto:info@hisparc.nl}{info@hisparc.nl}).


\chapter{Introduction}
\label{introduction:introduction}\label{introduction::doc}\label{introduction:welcome-to-the-hisparc-maintenance-documentation}
This documentation describes the steps required to solve problems with an HiSPARC station.  It first explains how to recognize different problems and then how to solve them.

The first part, {\hyperref[faq::doc]{\emph{Frequently asked questions}}}, might help answer some questions that you have.

The second part, {\hyperref[known-issues::doc]{\emph{Known issues}}}, describes some of the known issues that we have encountered with the HiSPARC software and hardware. It discribes how to determine if the problem described is what you are experiencing and then possible solutions.


\chapter{Frequently asked questions}
\label{faq::doc}\label{faq:frequently-asked-questions}\begin{itemize}
\item {} \begin{description}
\item[{Nothing works! What can I do?}] \leavevmode\begin{itemize}
\item {} 
Don't Panic! Try to determine where the problem lies and find the solution in this documentation.

\end{itemize}

\end{description}

\item {} \begin{description}
\item[{Where do I find the list of known possible issues?}] \leavevmode\begin{itemize}
\item {} 
Here: {\hyperref[known-issues::doc]{\emph{Known issues}}}

\end{itemize}

\end{description}

\item {} \begin{description}
\item[{The historgrams for my station don't look correct, how can I adjust the station settings to fix this?}] \leavevmode\begin{itemize}
\item {} 
Read here about \href{http://hisparc.github.com/station-software/doc/configuratie.html\#calibratie-van-de-detector}{calibrating the detectors}

\end{itemize}

\end{description}

\item {} \begin{description}
\item[{I tried that, I didn't find a similar problem! What do I do now?}] \leavevmode\begin{itemize}
\item {} 
Please contact your \href{http://www.hisparc.nl/over-hisparc/organisatie/}{cluster coordinator}  for further assitance. They may be aware of such problems the corresponding solutions to the problem, and if they are not, they will consult us.

\end{itemize}

\end{description}

\end{itemize}


\chapter{Known issues}
\label{known-issues:known-issues}\label{known-issues::doc}
This is a list of known possible issues with HiSPARC stations.
Each problem notes whether it has been fixed in a specific
version of the software and if it was perhaps introduced by one.
Also other parameters that could affect the occurance of the issue
are be noted (e.g. Windows XP/7, RAM, cable lengths).
Moreover, the Nagios warnings that an issue can trigger are mentioned.

\begin{notice}{note}{Note:}
Multiple issues can cause the same Nagios warning.
\end{notice}


\section{Structure}
\label{known-issues:structure}
Each problem has the following fields:
\begin{quote}\begin{description}
\item[{First Sign}] \leavevmode
Explaining how you will probably notice the problem.

\item[{Nagios}] \leavevmode
Nagios warnings that can be triggered.

\item[{Determination}] \leavevmode
This is a small guide explaining how to make sure that the problem being described is what you are experiencing.

\item[{Solution}] \leavevmode
How to solve it.

\item[{Effects}] \leavevmode
The effects of this problem.

\end{description}\end{quote}


\chapter{Software}
\label{known-issues:software}
This sections concerns itself with issues related to the HiSPARC
station-software.


\section{HiSPARC Monitor does not start}
\label{known-issues:hisparc-monitor-does-not-start}

\subsection{Missing directory}
\label{known-issues:missing-directory}\begin{quote}\begin{description}
\item[{First Sign}] \leavevmode
When the \code{STARTHiSPARCSoftware} program runs and the other programs (HiSPARC DAQ and Updater) start normally but the HiSPARC Monitor does not appear or closes instantly.

\item[{Nagios}] \leavevmode
\textbf{EventRate}, \textbf{StorageGrowth}, \textbf{StorageSize}, \textbf{TriggerRate}, and possibly \textbf{Buffer size}

\item[{Determination}] \leavevmode\begin{itemize}
\item {} 
Look in \code{hisparc/persistent/logs/src/} for the latest log file.

\item {} 
Check if there is a line that contains the text \code{Error: unable to open database file}.

\item {} 
Look in the \code{hisparc/persistent/data/} directory for a \code{hsmonitor} folder.

\item {} 
If it does not exist than go to the solution, otherwise this is not the problem.

\end{itemize}

\item[{Solution}] \leavevmode
Create the missing \code{hsmonitor} directory in \code{hisparc/persistent/data}.

\item[{Effects}] \leavevmode
The missing directory causes the HiSPARC Monitor to be unable to store events in its SQLite database, preventing it from sending events to the Nikhef datastore. Note that the HiSPARC DAQ should be unaffected. No events should be lost, the DAQ will store events in its MySQL database until the hard disc fills up.

\end{description}\end{quote}


\section{Hard Disc Space}
\label{known-issues:hard-disc-space}

\subsection{To many logs}
\label{known-issues:to-many-logs}\begin{quote}\begin{description}
\item[{First Sign}] \leavevmode
Nagios warning about Disc Space.

\item[{Nagios}] \leavevmode
\textbf{Disc Space}.

\item[{Determination}] \leavevmode\begin{itemize}
\item {} 
Look in hisparc/persistent/logs/.

\item {} 
Check the size of the src directory by right-clicking on it and choosing `Properties'.

\item {} 
Check if this is a significant fraction of the total disc space.

\end{itemize}

\item[{Solution}] \leavevmode
Remove all logs from the src directory except for the current one (present date in \code{dd-mm-yyyy.log}). Select all (ctrl + a) logs in \code{hisparc/persistent/logs/src}. Deselect the current one (ctrl + click). Remove them using shift + delete (to bypass the Recycle Bin)

\item[{Effects}] \leavevmode
If the disc is full the HiSPARC daq can not store events in the database, preventing the station from storing more events.

\end{description}\end{quote}


\subsection{To many updaters}
\label{known-issues:to-many-updaters}\begin{quote}\begin{description}
\item[{First Sign}] \leavevmode
Nagios warning about Disc Space.

\item[{Nagios}] \leavevmode
\textbf{Disc Space}

\item[{Determination}] \leavevmode\begin{itemize}
\item {} 
Look in hisparc/persistent/downloads/.

\item {} 
There should be some adminUpdater\_v\#\#.zip and userUnpacker\_v\#\#.exe files there.

\item {} 
By right-clicking them you can see their file size is of the order of 100 MB.

\item {} 
If there are many they can take up some space.

\end{itemize}

\item[{Solution}] \leavevmode
Remove all userUnpacker and adminUpdater files except the newest ones. Do this by selecting them and pressing shift + delete to remove them directly.

\item[{Effects}] \leavevmode
If the disc is full the HiSPARC daq can not store events in the database, preventing the station from storing more events.

\end{description}\end{quote}


\section{HiSPARC DAQ Errors}
\label{known-issues:hisparc-daq-errors}

\subsection{Can not connect to buffer}
\label{known-issues:can-not-connect-to-buffer}\begin{quote}\begin{description}
\item[{First Sign}] \leavevmode
Red LED in HiSPARC DAQ

\item[{Nagios}] \leavevmode
\item[{Determination}] \leavevmode
From the Start menu start odbcad32.exe. Check if the hisparc buffer is there.

\item[{Solution}] \leavevmode
\item[{Effects}] \leavevmode
\end{description}\end{quote}


\subsection{Not in DAQ Mode}
\label{known-issues:not-in-daq-mode}\begin{quote}\begin{description}
\item[{First Sign}] \leavevmode
\item[{Nagios}] \leavevmode
\textbf{TriggerRate}

\item[{Determination}] \leavevmode
Look at the program HiSPARC DAQ, see if the button in the middle shows `DAQ Mode'.

\item[{Solution}] \leavevmode
Click the `DAQ Mode' button in the HiSPARC DAQ.

\item[{Effects}] \leavevmode
When the HiSPARC DAQ is not in DAQ Mode it will not store triggered events.

\end{description}\end{quote}


\section{Error in HiSPARC Monitor}
\label{known-issues:error-in-hisparc-monitor}

\subsection{Malformed HisparcII.ini}
\label{known-issues:malformed-hisparcii-ini}\begin{quote}\begin{description}
\item[{First Sign}] \leavevmode
Errors in the HiSPARC Monitor: \code{Uncatched exception in job: need more than 1 value to unpack. Restarting...}

\item[{Nagios}] \leavevmode
\textbf{TriggerRate}

\item[{Determination}] \leavevmode
Check in the file hisparc/persistent/configuration/HisparcII.ini for blank lines

\item[{Solution}] \leavevmode
Remove any blank lines from HisparcII.ini

\item[{Effects}] \leavevmode
Errors in the HiSPARC Monitor and no TriggerRate updates for Nagios.

\end{description}\end{quote}


\subsection{Time difference to large}
\label{known-issues:time-difference-to-large}\begin{quote}\begin{description}
\item[{First Sign}] \leavevmode
Errors in the HiSPARC Monitor: \code{Uncatched exception in job: invalid literal for int() with base 10: 'difference too large'. Restarting...}

\item[{Nagios}] \leavevmode
\textbf{TriggerRate}

\item[{Determination}] \leavevmode
Check in the file hisparc/persistent/configuration/HisparcII.ini for the text `difference to large'.

\item[{Solution}] \leavevmode
Check the PC time, make sure that it is set to the current time. Check the GPS settings, make sure that it is working and showing the correct GPS time.

\item[{Effects}] \leavevmode
Errors in the HiSPARC Monitor and no TriggerRate updates for Nagios.

\end{description}\end{quote}


\section{GPS}
\label{known-issues:gps}

\subsection{COM Port to high}
\label{known-issues:com-port-to-high}\begin{quote}\begin{description}
\item[{First Sign}] \leavevmode
...

\item[{Nagios}] \leavevmode
\item[{Determination}] \leavevmode\begin{itemize}
\item {} 
Open Configuration -\textgreater{} System -\textgreater{} Hardware -\textgreater{} Browse Devices -\textgreater{} Com Ports

\item {} 
If the Com ports number higher than 32 the DSP Mon GPS program can not connect to the GPS

\end{itemize}

\item[{Solution}] \leavevmode
Lower the COM Port Number, by direct reassignment, use the com\_port\_reassign utility.

\item[{Effects}] \leavevmode
no GPS...

\end{description}\end{quote}


\chapter{Hardware}
\label{known-issues:hardware}


\renewcommand{\indexname}{Index}
\printindex
\end{document}
